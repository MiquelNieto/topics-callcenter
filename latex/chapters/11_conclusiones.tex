\chapter{Conclusiones}
\label{chapter:conclusiones}


\section{Boceto conclusiones}

Positivas: 

Es posible clasificar las llamadas atendiendo a dos visiones.  Supervisada y no supervisada. 

La importancia y dificultad de trabajar con datos reales. 

NEgativas: 

calidad de las etiquetas. 

Ciclo devops completo. 




\section{Aplicación aprendido}

Aunque se ha profundizado en algunos aspectos, más allá de lo aprendido a lo largo del máster, las asignaturas cursadas a lo largo de tres semestres han creado una base solida que han permitido el desarrollo de este proyecto.

En esta sección queremos hacer un repaso sobre la aplicación que todas las asignaturas cursadas a lo largo del máster, en mayor o menor medida, han tenido en este proyecto. 

\begin{itemize}

\item \textbf{Fundamentos de la ciencia de datos}: Esta asignatura nos ha permitido entender los principios básicos del mundo de la ciencia de datos. Entre otros aspectos la hemos utilizado para entender el ciclo de vida de un proyecto de ciencia de datos. También se han tratado otros aspectos que nos han sido útiles en el desarrollo del proyecto, como son la metodología \textit{Agile}, la calidad de los datos, etc.

\item \textbf{Tipología y ciclo de vida de los datos}: Esta asignatura es fundamental para entender el ciclo de vida de los datos. En ella hemos tratado los diferentes tipos de datos que podemos encontrar y las formas que tenemos de obtenerlos.

\item \textbf{Arquitecturas de bases de datos no tradicionales}: En esta asignatura se trabajó con diferentes  modelos de bases de datos NoSQL que han sido fundamentales para el desarrollo del proyecto. En nuestro proyecto hemos trabajados con \textit{Elasticsearch} en la capa de servicio, que aunque se trate de un motor de búsqueda presenta muchas similitudes con una BBDD NoSQL documental. Además se ha trabajado en la capa de \textit{streaming} con Apache Kafka que interiormente posee una BBDD RocksDB, que se trata de una NoSQL clave-valor. 

\item \textbf{Estadística avanzada}: Aunque no se han aplicado los modelos estadísticos vistos en esta asignatura, han  sido útiles los conceptos vistos para las fases de preprocesado y análisis de los datos.



\item \textbf{Minería de datos}: En esta asignatura se trabajaron a nivel práctico los métodos \textit{core} de la minería de datos. Para nuestro proyecto han sido de utilidad conceptos de preparación de datos, de \textit{clustering} y de evaluación de modelos. 

\item \textbf{Modelos avanzados de minería de datos}: En esta asignatura se profundiza más en los modelos de minería de datos y empezamos a aplicar las redes neuronales, que son la base de nuestros modelos supervisados, además tratamos temas que nos han sido útiles como la combinación de clasificadores.


\item  \textbf{Deep learning}: El contenido de esta asignatura ha sido vital para los modelos de aprendizaje supervisado. En ella hemos profundizado en el mundo de las redes neuronales y los diferentes tipos: convolucionales, recurrentes, etc.

\item  \textbf{Análisis de sentimientos y redes sociales}: En esta asignatura se trabajaron las bases del procesamiento del lenguaje natural que ha sido el \textit{core} de nuestro proyecto.

\item  \textbf{Visualización de datos}: en esta asignatura trabajamos los conceptos que existen detrás de una buena visualización. Estos conceptos se han intentado aplicar en nuestro proyecto, tanto en la fase de análisis como en la capa de servicio y visualización. Las visualizaciones interactivas se han creado en nuestro caso a través de Kibana.

\item  \textbf{Diseño y construcción del data warehouse}: Aunque no se han aplicado directamente los conocimientos de esta asignatura en el desarrollo del proyecto, sí ha sido necesario acceder al \textit{datawarehouse} de la empresa para la obtención de datos.


\item \textbf{Análisis de datos en entornos big data}: Aunque no hemos cursado esta asignatura, sí hemos aplicado algunos conceptos tratados en ella y relacionados con el uso de un ecosistema Hadoop. Principalmente HDFS, Spark y Hive.


\end{itemize}





\section{Líneas de trabajo futuras}

Una vez finalizado el proyecto y extraídas las conclusiones es el momento de abrir líneas de trabajos futuras que le den continuidad al trabajo aquí expuesto, ya sea ampliando su alcance o mejorando algunos aspectos.

\begin{itemize}
	\item Métricas Optimización: Modificar las 
	\item Etiquetas monitorización: Como hemos visto a lo largo del documento, principalmente en el capítulo \ref{chapter:dataset}, las etiquetas que tienen más calidad son las de monitorización, aunque tenían el problema del escaso porcentaje de etiquetas de este tipo. En un futuro se espera que el porcentaje de llamadas etiquetadas de este tipo aumente, pudiéndose aplicar los modelos vistos a estos datos.
	\item Etiquetas de sistemas operacionales: Una  opción que no se ha abordado hasta ahora es la de etiquetar las llamadas en función de las operaciones hechas por los clientes tras las llamadas: Si han realizado un alta, si han realizado una baja, si han abierto una reclamación, etc. Estas acciones pueden consultarse directamente en los sistemas operacionales de la empresa.
	\item Llevar a producción modelo no supervisado: Aunque se ha realizado un estudio no supervisado de las llamadas, este no se ha llevado a un entorno productivo. Puede ser muy interesante introducir un modelo no supervisado en nuestra arquitectura para tener en la capa de servicio una versión distinta a la de las etiquetas actuales. 
	\item Adaptar llamadas tiempo real, eliminar inyector: En este proyecto se ha presentado un elemento que simula las llamadas en tiempo real, a corto plazo será necesario eliminar este elemento y adaptar los datos de entrada del servicio \textit{tokenizer} a los datos reales de las llamadas. 
	\item Otras funcionalidades: Tanto la arquitectura expuesta en la capa de \textit{streaming}, como la capa de servicio pueden ser usadas para dotar al sistema de otras capacidades relacionadas con el modelo. Algunas de las funcionalidades que pueden ser útiles para complementar el sistema son: el análisis de sentimientos, analizar si una llamada ha supuesto una venta positiva, etc. 
\end{itemize}

\section{Caso de negocio}
Lo que me contó Nacho ver si encajarlo aquí o en la introducción. 

\section{Agradecimientos}



Un \textit{handicap} a la hora de realizar el proyecto dentro de una gran empresa ha sido el hecho de trabajar con unos plazos tan ajustados. Aspectos como la autorización en el acceso a la información, el acceso a diferentes entornos, la intercomunicación entre áreas, etc. requieren unos tiempos que pueden retrasar la ejecución de un trabajo de este tipo. 
 
Esta última sección tiene como objetivo agradecer a todas las personas que han hecho posible la realización de este proyecto en tiempo y forma.


A Willy Gavilán y Carolina Bouvard por hacer posible la  realización de este proyecto dentro de Telefónica. A Antonio Fernández por aceptar que realice este trabajo con su equipo y por tutorizarlo. A Jorge Ayuso y Rus Mesas por su ayuda a la hora de obtener los datos, por sus consejos y por permitirme usar con ellos el equipo con GPUs usados para  el entrenamiento y optimización de modelos. A Laura Canga por permitirme usar el Bus Apache Kafka y a José Ramón Fernández por ayudarme, incluso a deshoras, con la creación de \textit{topics} y usuarios del bus.  A Pablo Palomares Darocas por su ayuda a la hora de acceder y adaptar a nuestras necesidades las plataformas \textit{DevOps}: Bitbucket, Nexus y Jenkins. A Nacho Charfolé por darme su visión más allá de la parte técnica y por hacerme entender las ventajas que pueden aportar este tipo de proyectos a una empresa. Y por último, pero no menos importante, a mi tutora, Ana Valdivia, por sus concreciones minuciosas y por guiarme y animarme en todo momento.

Sin todos vosotros un trabajo de este tipo no sería posible. \textbf{Gracias}.



 