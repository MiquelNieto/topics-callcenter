\chapter{Conclusiones}
\label{chapter:conclusiones}

\section{Aplicación aprendido}

Aunque se ha profundizado en algunos aspectos, más allá de lo aprendido a lo largo del máster, las asignaturas cursadas a lo largo de tres semestres han creado una base solida que han permitido el desarrollo de este proyecto.

En esta sección queremos hacer un repaso sobre la aplicación que todas las asignaturas cursadas a lo largo del máster, en mayor o menor medida, han tenido en este proyecto. 

\begin{itemize}

\item \textbf{Fundamentos de la ciencia de datos}: Esta asignatura nos ha permitido entender los principios básicos del mundo de la ciencia de datos. Entre otros aspectos la hemos utilizado para entender el ciclo de vida de un proyecto de ciencia de datos. También se han tratado otros aspectos que nos han sido útiles en el desarrollo del proyecto, como son la metodología \textit{Agile}, la calidad de los datos, etc.

\item \textbf{Tipología y ciclo de vida de los datos}: Esta asignatura es fundamental para entender el ciclo de vida de los datos. En ella hemos tratado los diferentes tipos de datos que podemos encontrar y las formas que tenemos de obtenerlos.

\item \textbf{Arquitecturas de bases de datos no tradicionales}: En esta asignatura se trabajó con diferentes  modelos de bases de datos NoSQL que han sido fundamentales para el desarrollo del proyecto. En nuestro proyecto hemos trabajados con \textit{Elasticsearch} en la capa de servicio, que aunque se trate de un motor de búsqueda presenta muchas similitudes con una BBDD NoSQL documental. Además se ha trabajado en la capa de \textit{streaming} con Apache Kafka que interiormente posee una BBDD RocksDB, que se trata de una NoSQL clave-valor. 

\item \textbf{Estadística avanzada}: Aunque no se han aplicado los modelos estadísticos vistos en esta asignatura, han  sido útiles los conceptos vistos para las fases de preprocesado y análisis de los datos.



\item \textbf{Minería de datos}: En esta asignatura se trabajaron a nivel práctico los métodos \textit{core} de la minería de datos. Para nuestro proyecto han sido de utilidad conceptos de preparación de datos, de \textit{clustering} y de evaluación de modelos. 

\item \textbf{Modelos avanzados de minería de datos}: En esta asignatura se profundiza más en los modelos de minería de datos y empezamos a aplicar las redes neuronales, que son la base de nuestros modelos supervisados, además tratamos temas que nos han sido útiles como la combinación de clasificadores.


\item  \textbf{Deep learning}: El contenido de esta asignatura ha sido vital para los modelos de aprendizaje supervisado. En ella hemos profundizado en el mundo de las redes neuronales y los diferentes tipos: convolucionales, recurrentes, etc.

\item  \textbf{Análisis de sentimientos y redes sociales}: En esta asignatura se trabajaron las bases del procesamiento del lenguaje natural que ha sido el \textit{core} de nuestro proyecto.

\item  \textbf{Visualización de datos}: en esta asignatura trabajamos los conceptos que existen detrás de una buena visualización. Estos conceptos se han intentado aplicar en nuestro proyecto, tanto en la fase de análisis como en la capa de servicio y visualización. Las visualizaciones interactivas se han creado en nuestro caso a través de Kibana.

\item  \textbf{Diseño y construcción del data warehouse}: Aunque no se han aplicado directamente los conocimientos de esta asignatura en el desarrollo del proyecto, sí ha sido necesario acceder al \textit{datawarehouse} de la empresa para la obtención de datos.


\item \textbf{Análisis de datos en entornos big data}: Aunque no hemos cursado esta asignatura, sí hemos aplicado algunos conceptos tratados en ella y relacionados con el uso de un ecosistema Hadoop. Principalmente HDFS, Spark y Hive.


\end{itemize}





\section{Líneas de trabajo futuras}

\begin{itemize}
	\item Métricas Optimización: 
	\item Etiquetas monitorización: 
	\item Profundizar estudio no supervisado: 
	\item Llevar a producción modelo no supervisado. 
	\item Adaptar llamadas tiempo real, eliminar inyector. 
	\item Disponer llamadas 
\end{itemize}

\section{Caso de negocio}
Lo que me contó Nacho ver si encajarlo aquí o en la introducción. 

\section{Agradecimientos}

Un \textit{handicap} a la hora de realizar el proyecto dentro de una gran empresa ha sido el hecho de trabajar con unos plazos tan ajustados. Aspectos como la autorización en el acceso a la información, el acceso a diferentes entornos, la intercomunicación entre áreas, etc. requieren unos tiempos que pueden retrasar la ejecución de un trabajo de este tipo. 
 
Esta última sección tiene como objetivo dar las gracias a todas las personas que han hecho posible la realización de este proyecto en tiempo y forma.

A Antonio Fernández Gallardo

A Willy Gavilán Montegro y Carolina Bouvard Nuño por permitirme realizar el proyecto dentro de Telefónica.

A Jorge Ayuso Rejas y Rus Mesas Javega por su ayuda a la hora de obtener los datos y por permitirme usar con ellos el equipo con GPUs usados por el entrenamiento. 

A Laura Canga García por dejarme usar el Bus Kafka y a José Ramón Fernández Acosta por ayudarme, incluso a deshoras, con la creación y usuarios del bus. 

A Pablo Palomares Darocas por su ayuda a la hora de acceder y usar las plataformas \textit{DevOps}: Bitbucket, Nexus y Jenkins.

A Nacho Charfolé Sancho por darme su visión más allá de la parte técnica.

 